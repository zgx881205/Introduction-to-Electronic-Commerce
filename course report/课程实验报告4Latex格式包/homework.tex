\documentclass{article}

%
% 引入模板的style文件
%
\usepackage{url}
\usepackage{homework}
\usepackage{enumerate}
\usepackage{amsmath}
%
% 封面
%

\title{
	\includegraphics[scale = 0.45]{images/title/ucas-logo1.png}\\
    \vspace{1in}
    \textmd{\textbf{\hmwkClass\ \hmwkTitle}}\\
    \textmd{\textbf{\hmwkSubTitle}}\\
    \normalsize\vspace{0.1in}\small{\hmwkCompleteTime }\\
    \vspace{0.1in}\large{\textit{\hmwkClassInstructor\ }}\\
    \vspace{3in}
}

\author{\hmwkAuthorName \\ 
	\hmwkAuthorStuID}
\date{}

\renewcommand{\part}[1]{\textbf{\large Part \Alph{partCounter}}\stepcounter{partCounter}\\}


%
% 正文部分
%
\begin{document}


\maketitle


%\include{chapters/ch01}


\pagebreak

\begin{homeworkProblem}
 \noindent\textbf{1.什么的协同过滤(Collaborative Filtering)推荐算法?分别阐释基于用户的协同过滤算法(User-based Collaborative Filtering, UCF)~\cite{G_UCF}和基于项目的协同过滤算法(Item-based Collaborative Filtering, ICF)~\cite{linden2003amazon}的原理。}\\
\textbf{Solution:}
{\color{blue}
\begin{itemize}
	\item [(1)]协同过滤:
	\item [(2)]UCF:
	\item [(3)]ICF:
\end{itemize}

}
\end{homeworkProblem}

\pagebreak


\begin{homeworkProblem}
 \noindent\textbf{2.使用Python中的scikit-surprise库~\cite{Hug2020}加载电影评分数据集``ml-100k",并采用库中的$train\_test\_split$函数以4:1的比例划分训练集和测试集,分别使用UCF和ICF在训练集上进行模型的训练,在测试集采用MAE的指标对模型进行评价,同时输出测试集中每一个用户Top-5的电影推荐列表。}\\
\textbf{Solution:}
{\color{blue} 
\begin{itemize}
	\item [(1)]UCF:
	\item [(2)]ICF:
\end{itemize}
}


\end{homeworkProblem}

% 引用文献
\bibliographystyle{unsrt}  % unsrt:根据引用顺序编号
\bibliography{refs}


\end{document}
