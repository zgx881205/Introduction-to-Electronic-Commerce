\documentclass{article}

%
% 引入模板的style文件
%
\usepackage{url}
\usepackage{homework}


%
% 封面
%

\title{
	\includegraphics[scale = 0.45]{images/title/ucas-logo1.png}\\
    \vspace{1in}
    \textmd{\textbf{\hmwkClass\ \hmwkTitle}}\\
    \textmd{\textbf{\hmwkSubTitle}}\\
    \normalsize\vspace{0.1in}\small{\hmwkCompleteTime }\\
    \vspace{0.1in}\large{\textit{\hmwkClassInstructor\ }}\\
    \vspace{3in}
}

\author{\hmwkAuthorName \\ 
	\hmwkAuthorStuID}
\date{}

\renewcommand{\part}[1]{\textbf{\large Part \Alph{partCounter}}\stepcounter{partCounter}\\}


%
% 正文部分
%
\begin{document}


\maketitle


%\include{chapters/ch01}


\pagebreak

\begin{homeworkProblem}
\textbf{1.	通过Python中Scikit-Learn\cite{pedregosa2011scikit}载入名为load\_wine(红酒)数据集,选用Scikit-Learn中任意一种分类器进行训练集和测试集的划分,并利用训练集的样本对分类器进行训练,最终在测试集上进行分类预测,输出模型在测试集上的准确率(代码采用截图方式呈现)。}\\
\textbf{Solution:}\\

{\color{blue}(1)载入数据


(2)划分训练集和测试集



(3)在训练集上训练分类器


(4)在测试集上进行分类,计算分类器的准确率  





}
\end{homeworkProblem}

\pagebreak

\begin{homeworkProblem}
	\textbf{2.简述K-Means算法\cite{likas2003global}的原理,以及模型的输入、输出及聚类过程(流程)。}\\
	\textbf{Solution:}\\

\end{homeworkProblem}


\pagebreak

\begin{homeworkProblem}
\textbf{附加题. 使用Python中的numpy库生成一组随机数据:样本大小为100, 特征数为3,采用Scikit-Learn库中K-Means算法进行聚类,并对聚类结果进行可视化展示。}\\
\textbf{Solution:}\\
{\color{blue}


}


\end{homeworkProblem}

% 引用文献
\bibliographystyle{unsrt}  % unsrt:根据引用顺序编号
\bibliography{refs}


\end{document}
